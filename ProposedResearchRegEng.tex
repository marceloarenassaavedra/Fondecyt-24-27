\documentclass[10pt]{article}

%!TEX root = ../main/main.tex

%%%%%%%%%%%%%%%%%%%%%%%%%%%%%%%%
% extras/packages.tex
%%%%%%%%%%%%%%%%%%%%%%%%%%%%%%%%

\usepackage[utf8]{inputenc}
\usepackage{makeidx}
\usepackage{multirow}
\usepackage{multicol}
\usepackage[dvipsnames,svgnames,table]{xcolor}
\usepackage{graphicx}
\usepackage{epstopdf}
%\usepackage{ulem}
\usepackage{hyperref}
\usepackage{amsmath}
\usepackage{amssymb}
%\usepackage[paperwidth=612pt,paperheight=792pt,top=28pt,right=34pt,bottom=35pt,left=34pt]{geometry}
\usepackage[top=40pt,right=60pt,bottom=50pt,left=60pt]{geometry}

%\author{Felipe Vásquez Moraga}
%\title{PROPOSED RESEARCH REGULAR 2019}

\begin{document}

\noindent
\textbf{Research Proposal} \\

{\raggedright

\vspace{3pt} \noindent
\begin{tabular}{|p{\textwidth}|}
\hline
\parbox{\textwidth} {\raggedright  \vspace{3pt}
In this section, you must present your proposal by developing the following aspects: a) Theoretical-conceptual foundations and state of the art that support the proposal, b) Hypothesis or research questions and objectives, c) Scientific or technological novelty of your proposal, d) Methodology, e) Work plan or Gantt chart and f) Background information to assess the capacity of the team to implement the proposal.

\vspace{2mm}

Remember:
\begin{itemize}
\item[-] You must strictly comply with what is established in Bases Concurso Nacional de Proyectos Fondecyt Regular 2024.
\item[-] All text, paragraphs, or textual phrases from a bibliographic reference, whether by other authors or their own, must be duly identified in the text and in the list of references.
\end{itemize}

This file must contain a maximum of {\bf 10 pages} (use letter size format, Verdana font size 10 or similar).

\vspace{2mm}

Aspects related to the proposed research that is included in annexes will not be considered in the evaluation process.}\\
\hline
\end{tabular}
\vspace{2pt}

}

\section{Theoretical-conceptual foundations and state of the art}
\section{Research questions and objectives}
\begin{itemize}
    \item How to control the source of information used by LLM
    \item How to increase the confidence in LLM answers while preserving their ease of use.
    \item How to reduce the cost of using LLM
\end{itemize}

\section{Scientific and technological novelty}

\begin{itemize}
    \item DBMS specialized in logical plans for enumeration based on semantic relevancy.
    \item Indexing strategy with Metadata/type.
    \item graph-embeddings for classification/navigation
\end{itemize}

\section{Methodology}
\begin{itemize}
    \item Graph because there is straightforward textualization and existing embedding techniques.
    \item Implementation parralele to research (BUSEA+MilliniumDB/AI). \textit{Important due to the assumption about base-enumerators asymptotic behaviors.}
\end{itemize}



\section{Work plan}
\section{Background information}



\newpage

\noindent
\textbf{BIBLIOGRAPHIC REFERENCES:} \\
In this section, include the
complete list of cited references in the Proposed Research
section. {\bf Maximum extension 5 pages.} (Must use letter size, Verdana
size 10 or similar).
\\
\\


\end{document}
